\section*{Annex}

\subsection{Detailed example of a failed execution}
%TODO try to define a macro for figures

\paragraph{}
We're presenting here an example of an execution that fails with the algorithm
described by the paper. The input of the algorithm is as follows~:
\begin{itemize}
\item $G$ is described by the initial situation in figure~\ref{FE_K2_init}
\item $k = 2$
\item $\mathrm{roots} = \{ e, a \}$
\item $\mathrm{partitionSize} = \{ 5, 7\}$
\end{itemize}

\paragraph{}
In order to see easily from where the trees are starting and what vertices are
included in the built partition. Vertex and edges which are included in a tree
have a stronger thickness.

\begin{figure}[H]
  \caption{\label{FE_K2_init}Failed Execution : Initial graph}
  \begin{center}
    \begin{tikzpicture}[scale=0.9,transform shape]
      % Vertices
\SetVertexNormal[LineWidth=3pt]%V2
\Vertex[x=4.8,y=5  ]{a}
\SetVertexNormal[LineWidth=3pt]%V1
\Vertex[x=0  ,y=0  ]{e}
\SetVertexNormal[LineWidth=1pt]%Vfree
\Vertex[x=3.2,y=8.2]{b}
\Vertex[x=4.8,y=7  ]{c}
\Vertex[x=6.4,y=8.2]{d} 
\Vertex[x=3.2,y=0  ]{f}
\Vertex[x=1.6,y=3.2]{g}
\Vertex[x=1.6,y=1.2]{h}
\Vertex[x=6.4,y=0  ]{i}
\Vertex[x=8  ,y=3.2]{j}
\Vertex[x=9.6,y=0  ]{k}
\Vertex[x=8  ,y=1.2]{l}
%Cliques
\Edge[lw=1pt](a)(b)
\Edge[lw=1pt](a)(c)
\Edge[lw=1pt](a)(d)
\Edge[lw=1pt](b)(c)
\Edge[lw=1pt](b)(d)
\Edge[lw=1pt](c)(d)
\Edge[lw=1pt](e)(f)
\Edge[lw=1pt](e)(g)
\Edge[lw=1pt](e)(h)
\Edge[lw=1pt](f)(g)
\Edge[lw=1pt](f)(h)
\Edge[lw=1pt](g)(h)
\Edge[lw=1pt](i)(j)
\Edge[lw=1pt](i)(k)
\Edge[lw=1pt](i)(l)
\Edge[lw=1pt](j)(k)
\Edge[lw=1pt](j)(l)
\Edge[lw=1pt](k)(l)
%Cliques Connection
\Edge[lw=1pt](a)(f)
\Edge[lw=1pt](b)(g)
\Edge[lw=1pt](f)(i)
\Edge[lw=1pt](g)(j)

    \end{tikzpicture}
  \end{center}
\end{figure}

\paragraph{}
During the first two steps, each partition needs to acquire many vertices,
each one will grow by adding all the available neighborhood.

\begin{figure}[H]
  \caption{Failed Execution : Step 1}
  \begin{center}
    \begin{tikzpicture}[scale=0.9,transform shape]
      % Vertices
\SetVertexNormal[MinSize=25pt,LineColor=violet,LineWidth=3pt]%V2
\Vertex[x= 2,y=0]{a}
\SetVertexNormal[MinSize=25pt,LineColor=orange,LineWidth=3pt]%V1
\Vertex[x= 6,y=2]{e}
\Vertex[x= 8,y=0]{f}
\Vertex[x= 8,y=4]{g}
\Vertex[x=10,y=2]{h}
\SetVertexNormal[MinSize=25pt,LineColor=black,LineWidth=3pt]%Vfree
\Vertex[x= 2,y=4]{b}
\Vertex[x= 4,y=2]{c}
\Vertex[x= 0,y=2]{d}
\Vertex[x=14,y=0]{i}
\Vertex[x=14,y=4]{j}
\Vertex[x=12,y=2]{k}
\Vertex[x=16,y=2]{l}
%Cliques
\Edge[lw=1pt,color=black](a)(b)%Efree
\Edge[lw=1pt,color=black](a)(c)%Efree
\Edge[lw=1pt,color=black](a)(d)%Efree
\Edge[lw=1pt,color=black](b)(c)%Efree
\Edge[lw=1pt,color=black](b)(d)%Efree
\Edge[lw=1pt,color=black](c)(d)%Efree
\Edge[lw=3pt,color=orange](e)(f)%E1
\Edge[lw=3pt,color=orange](e)(g)%E1
\Edge[lw=3pt,color=orange](e)(h)%E1
\Edge[lw=1pt,color=black](f)(g)%Efree
\Edge[lw=1pt,color=black](f)(h)%Efree
\Edge[lw=1pt,color=black](g)(h)%Efree
\Edge[lw=1pt,color=black](i)(j)%Efree
\Edge[lw=1pt,color=black](i)(k)%Efree
\Edge[lw=1pt,color=black](i)(l)%Efree
\Edge[lw=1pt,color=black](j)(k)%Efree
\Edge[lw=1pt,color=black](j)(l)%Efree
\Edge[lw=1pt,color=black](k)(l)%Efree
%Cliques Connection
\Edge[lw=1pt,color=black](a)(f)%Efree
\Edge[lw=1pt,color=black](b)(g)%Efree
\Edge[lw=1pt,color=black](f)(i)%Efree
\Edge[lw=1pt,color=black](g)(j)%Efree

    \end{tikzpicture}
  \end{center}
\end{figure}

\begin{figure}[H]
  \caption{Failed Execution : Step 2}
  \begin{center}
    \begin{tikzpicture}[scale=0.9,transform shape]
      % Vertices
\Vertex[x=7,y=5]{a} 
\Vertex[x=5,y=7]{b}
\Vertex[x=7,y=9]{c}
\Vertex[x=9,y=7]{d} 
\Vertex[x=0,y=2]{e}
\Vertex[x=2,y=4]{f}
\Vertex[x=4,y=2]{g}
\Vertex[x=2,y=0]{h}
\Vertex[x=10,y=2]{i}
\Vertex[x=12,y=4]{j}
\Vertex[x=14,y=2]{k}
\Vertex[x=12,y=0]{l}
%Cliques
\Edge[lw=3pt](a)(b)
\Edge[lw=3pt](a)(c)
\Edge[lw=3pt](a)(d)
\Edge[lw=1pt](b)(c)
\Edge[lw=1pt](b)(d)
\Edge[lw=1pt](c)(d)
\Edge[lw=3pt](e)(f)
\Edge[lw=3pt](e)(g)
\Edge[lw=3pt](e)(h)
\Edge[lw=1pt](f)(g)
\Edge[lw=1pt](f)(h)
\Edge[lw=1pt](g)(h)
\Edge[lw=1pt](i)(j)
\Edge[lw=1pt](i)(k)
\Edge[lw=1pt](i)(l)
\Edge[lw=1pt](j)(k)
\Edge[lw=1pt](j)(l)
\Edge[lw=1pt](k)(l)
%Cliques Connection
\Edge[lw=1pt](a)(f)
\Edge[lw=1pt](b)(g)
\Edge[lw=1pt](f)(i)
\Edge[lw=1pt](g)(j)

    \end{tikzpicture}
  \end{center}
\end{figure}

\paragraph{}
At step 3, two vertices will be available to increase the size of the current
working tree: $i$ and $j$. Since it needs only one vertex more, it will be
one of these two randomly, in this case, the vertex $j$ is choosen.

\begin{figure}[H]
  \caption{Failed Execution : Step 3}
  \begin{center}
    \begin{tikzpicture}[scale=0.9,transform shape]
      % Vertices
\SetVertexNormal[MinSize=25pt,LineColor=violet,LineWidth=3pt]%V2
\Vertex[x= 2,y=0]{a} 
\Vertex[x= 2,y=4]{b}
\Vertex[x= 4,y=2]{c}
\Vertex[x= 0,y=2]{d}
\SetVertexNormal[MinSize=25pt,LineColor=orange,LineWidth=3pt]%V1
\Vertex[x= 6,y=2]{e}
\Vertex[x= 8,y=0]{f}
\Vertex[x= 8,y=4]{g}
\Vertex[x=10,y=2]{h}
\Vertex[x=14,y=4]{j}
\SetVertexNormal[MinSize=25pt,LineColor=black,LineWidth=3pt]%Vfree
\Vertex[x=14,y=0]{i}
\Vertex[x=12,y=2]{k}
\Vertex[x=16,y=2]{l}
%Cliques
\Edge[lw=3pt,color=violet](a)(b)%E2
\Edge[lw=3pt,color=violet](a)(c)%E2
\Edge[lw=3pt,color=violet](a)(d)%E2
\Edge[lw=1pt,color=black](b)(c)%Efree
\Edge[lw=1pt,color=black](b)(d)%Efree
\Edge[lw=1pt,color=black](c)(d)%Efree
\Edge[lw=3pt,color=orange](e)(f)%E1
\Edge[lw=3pt,color=orange](e)(g)%E1
\Edge[lw=3pt,color=orange](e)(h)%E1
\Edge[lw=1pt,color=black](f)(g)%Efree
\Edge[lw=1pt,color=black](f)(h)%Efree
\Edge[lw=1pt,color=black](g)(h)%Efree
\Edge[lw=1pt,color=black](i)(j)%Efree
\Edge[lw=1pt,color=black](i)(k)%Efree
\Edge[lw=1pt,color=black](i)(l)%Efree
\Edge[lw=1pt,color=black](j)(k)%Efree
\Edge[lw=1pt,color=black](j)(l)%Efree
\Edge[lw=1pt,color=black](k)(l)%Efree
%Cliques Connection
\Edge[lw=1pt,color=black](a)(f)%Efree
\Edge[lw=1pt,color=black](b)(g)%Efree
\Edge[lw=1pt,color=black](f)(i)%Efree
\Edge[lw=3pt,color=orange](g)(j)%E1

    \end{tikzpicture}
  \end{center}
\end{figure}

\paragraph{}
At step 4, the tree rooted at $a$ can't be simply extended, since every
adjacent vertex belong to the other tree. The adjacent vertices are $f$ and
$g$. Both respects the condition necessary to allows the swaping. According to
the algorithm, the vertex to swap must have the lowest degree in the concerned
tree among the swapping candidates. It can't be $f$ since it has a higher
internal degree than $g$ in the concerned. The swapped vertex will then be
$f$.

\begin{figure}[H]
  \caption{Failed Execution : Step 4}
  \begin{center}
    \begin{tikzpicture}[scale=0.9,transform shape]
      % Vertices
\SetVertexNormal[MinSize=25pt,LineColor=violet,LineWidth=3pt]%V2
\Vertex[x=4.8,y=5  ]{a} 
\Vertex[x=3.2,y=8.2]{b}
\Vertex[x=4.8,y=7  ]{c}
\Vertex[x=6.4,y=8.2]{d}
\Vertex[x=3.2,y=0  ]{f} 
\SetVertexNormal[MinSize=25pt,LineColor=orange,LineWidth=3pt]%V1
\Vertex[x=0  ,y=0  ]{e}
\Vertex[x=1.6,y=3.2]{g}
\Vertex[x=1.6,y=1.2]{h}
\Vertex[x=8  ,y=3.2]{j}
\SetVertexNormal[MinSize=25pt,LineColor=black,LineWidth=1pt]%Vfree
\Vertex[x=6.4,y=0  ]{i}
\Vertex[x=9.6,y=0  ]{k}
\Vertex[x=8  ,y=1.2]{l}
%Cliques
\Edge[lw=3pt,color=violet](a)(b)%E2
\Edge[lw=3pt,color=violet](a)(c)%E2
\Edge[lw=3pt,color=violet](a)(d)%E2
\Edge[lw=1pt,color=black](b)(c)%Efree
\Edge[lw=1pt,color=black](b)(d)%Efree
\Edge[lw=1pt,color=black](c)(d)%Efree
\Edge[lw=1pt,color=black](e)(f)%Efree
\Edge[lw=3pt,color=orange](e)(g)%E1
\Edge[lw=3pt,color=orange](e)(h)%E1
\Edge[lw=1pt,color=black](f)(g)%Efree
\Edge[lw=1pt,color=black](f)(h)%Efree
\Edge[lw=1pt,color=black](g)(h)%Efree
\Edge[lw=1pt,color=black](i)(j)%Efree
\Edge[lw=1pt,color=black](i)(k)%Efree
\Edge[lw=1pt,color=black](i)(l)%Efree
\Edge[lw=1pt,color=black](j)(k)%Efree
\Edge[lw=1pt,color=black](j)(l)%Efree
\Edge[lw=1pt,color=black](k)(l)%Efree
%Cliques Connection
\Edge[lw=3pt,color=violet](a)(f)%E2
\Edge[lw=1pt,color=black](b)(g)%Efree
\Edge[lw=1pt,color=black](f)(i)%Efree
\Edge[lw=3pt,color=orange](g)(j)%E1

    \end{tikzpicture}
  \end{center}
\end{figure}

\paragraph{}
At step 5, the tree rooted at $e$ must get back a vertex, since the other tree
has taken one of it's vertices. The two candidates are $i$ and $k$. Since only
one is missing, only one must be added. In our case, $i$ will be added.

\begin{figure}[H]
  \caption{Failed Execution : Step 5}
  \begin{center}
    \begin{tikzpicture}[scale=0.9,transform shape]
      % Vertices
\SetVertexNormal[LineWidth=3pt]
\Vertex[x=7,y=5]{a} 
\Vertex[x=5,y=7]{b}
\Vertex[x=7,y=9]{c}
\Vertex[x=9,y=7]{d} 
\Vertex[x=0,y=2]{e}
\Vertex[x=2,y=4]{f}
\Vertex[x=4,y=2]{g}
\Vertex[x=2,y=0]{h}
\Vertex[x=10,y=2]{i}
\Vertex[x=12,y=4]{j}
\SetVertexNormal[LineWidth=1pt]
\Vertex[x=14,y=2]{k}
\Vertex[x=12,y=0]{l}
%Cliques
\Edge[lw=3pt](a)(b)
\Edge[lw=3pt](a)(c)
\Edge[lw=3pt](a)(d)
\Edge[lw=1pt](b)(c)
\Edge[lw=1pt](b)(d)
\Edge[lw=1pt](c)(d)
\Edge[lw=1pt](e)(f)
\Edge[lw=3pt](e)(g)
\Edge[lw=3pt](e)(h)
\Edge[lw=1pt](f)(g)
\Edge[lw=1pt](f)(h)
\Edge[lw=1pt](g)(h)
\Edge[lw=3pt](i)(j)
\Edge[lw=1pt](i)(k)
\Edge[lw=1pt](i)(l)
\Edge[lw=1pt](j)(k)
\Edge[lw=1pt](j)(l)
\Edge[lw=1pt](k)(l)
%Cliques Connection
\Edge[lw=3pt](a)(f)
\Edge[lw=1pt](b)(g)
\Edge[lw=1pt](f)(i)
\Edge[lw=3pt](g)(j)

    \end{tikzpicture}
  \end{center}
\end{figure}

\paragraph{}
At step 6, the tree rooted at $a$ has no adjacent available vertex, it has
then to swap a vertex among the neighborhood: $e$,$g$,$h$,$i$. $e$ is not an
option since it's the root of a partition. $g$ has an internal degree of 2, it
can't be choosen since $h$ and $i$ have an internal degree of 1. The swapped
vertex must be $h$ or $i$, since nothing in the algorithm determines which one
must be choosen, $h$ might be swapped at that step.

\begin{figure}[H]
  \caption{Failed Execution : Step 6}
  \begin{center}
    \begin{tikzpicture}[scale=0.9,transform shape]
      % Vertices
\SetVertexNormal[MinSize=25pt,LineColor=violet,LineWidth=3pt]%V2
\Vertex[x= 2,y=0]{a} 
\Vertex[x= 2,y=4]{b}
\Vertex[x= 4,y=2]{c}
\Vertex[x= 0,y=2]{d}
\Vertex[x= 8,y=0]{f}
\Vertex[x=10,y=2]{h} 
\SetVertexNormal[MinSize=25pt,LineColor=orange,LineWidth=3pt]%V1
\Vertex[x= 6,y=2]{e}
\Vertex[x= 8,y=4]{g}
\Vertex[x=14,y=0]{i}
\Vertex[x=14,y=4]{j}
\SetVertexNormal[MinSize=25pt,LineColor=black,LineWidth=3pt]%Vfree
\Vertex[x=12,y=2]{k}
\Vertex[x=16,y=2]{l}
%Cliques
\Edge[lw=3pt,color=violet](a)(b)%E2
\Edge[lw=3pt,color=violet](a)(c)%E2
\Edge[lw=3pt,color=violet](a)(d)%E2
\Edge[lw=1pt,color=black](b)(c)%Efree
\Edge[lw=1pt,color=black](b)(d)%Efree
\Edge[lw=1pt,color=black](c)(d)%Efree
\Edge[lw=1pt,color=black](e)(f)%Efree
\Edge[lw=3pt,color=orange](e)(g)%E1
\Edge[lw=1pt,color=black](e)(h)%Efree
\Edge[lw=1pt,color=black](f)(g)%Efree
\Edge[lw=3pt,color=violet](f)(h)%E2
\Edge[lw=1pt,color=black](g)(h)%Efree
\Edge[lw=3pt,color=orange](i)(j)%E1
\Edge[lw=1pt,color=black](i)(k)%Efree
\Edge[lw=1pt,color=black](i)(l)%Efree
\Edge[lw=1pt,color=black](j)(k)%Efree
\Edge[lw=1pt,color=black](j)(l)%Efree
\Edge[lw=1pt,color=black](k)(l)%Efree
%Cliques Connection
\Edge[lw=3pt,color=violet](a)(f)%E2
\Edge[lw=1pt,color=black](b)(g)%Efree
\Edge[lw=1pt,color=black](f)(i)%Efree
\Edge[lw=3pt,color=orange](g)(j)%E1

    \end{tikzpicture}
  \end{center}
\end{figure}

\paragraph{}
At step 7, the tree rooted at $e$ must add a vertex among $k$ and $l$, the
vertex $k$ can be freely added with the edge $(i,k)$.

\begin{figure}[H]
  \caption{Failed Execution : Step 7}
  \begin{center}
    \begin{tikzpicture}[scale=0.9,transform shape]
      % Vertices
\SetVertexNormal[LineWidth=3pt]%V2
\Vertex[x=4.8,y=5  ]{a} 
\Vertex[x=3.2,y=8.2]{b}
\Vertex[x=4.8,y=7  ]{c}
\Vertex[x=6.4,y=8.2]{d} 
\Vertex[x=3.2,y=0  ]{f}
\Vertex[x=1.6,y=1.2]{h}
\SetVertexNormal[LineWidth=3pt]%V1
\Vertex[x=0  ,y=0  ]{e}
\Vertex[x=1.6,y=3.2]{g}
\Vertex[x=6.4,y=0  ]{i}
\Vertex[x=8  ,y=3.2]{j}
\Vertex[x=9.6,y=0  ]{k}
\SetVertexNormal[LineWidth=1pt]%Vfree
\Vertex[x=8  ,y=1.2]{l}
%Cliques
\Edge[lw=3pt](a)(b)%E2
\Edge[lw=3pt](a)(c)%E2
\Edge[lw=3pt](a)(d)%E2
\Edge[lw=1pt](b)(c)%Efree
\Edge[lw=1pt](b)(d)%Efree
\Edge[lw=1pt](c)(d)%Efree
\Edge[lw=1pt](e)(f)%Efree
\Edge[lw=3pt](e)(g)%E1
\Edge[lw=1pt](e)(h)%Efree
\Edge[lw=1pt](f)(g)%Efree
\Edge[lw=3pt](f)(h)%E2
\Edge[lw=1pt](g)(h)%Efree
\Edge[lw=3pt](i)(j)%E1
\Edge[lw=3pt](i)(k)%E1
\Edge[lw=1pt](i)(l)%Efree
\Edge[lw=1pt](j)(k)%Efree
\Edge[lw=1pt](j)(l)%Efree
\Edge[lw=1pt](k)(l)%Efree
%Cliques Connection
\Edge[lw=3pt](a)(f)%E2
\Edge[lw=1pt](b)(g)%Efree
\Edge[lw=1pt](f)(i)%Efree
\Edge[lw=3pt](g)(j)%E1

    \end{tikzpicture}
  \end{center}
\end{figure}

\paragraph{}
At step 8, the tree rooted at $a$ must take a last vertex, it has no free
adjacent vertex, it must then take a vertex from the tree rooted at $e$. The
allowede vertices for a swapping are $g$ and $i$. Since they have both the
same internal degree in the concerned tree, $g$ can be choosen as a valid
choice for the extension.

\paragraph{}
By swapping with $g$, the subtree rooted at $g$ must be cut from the tree
rooted at $e$. All the removed vertices ($i$,$j$,$k$) will be set back as
unexplored, but not the swapped vertex, $g$.

\begin{figure}[H]
  \caption{Failed Execution : Step 8}
  \begin{center}
    \begin{tikzpicture}[scale=0.9,transform shape]
      % Vertices
\Vertex[x=7,y=5]{a} 
\Vertex[x=5,y=7]{b}
\Vertex[x=7,y=9]{c}
\Vertex[x=9,y=7]{d} 
\Vertex[x=0,y=2]{e}
\Vertex[x=2,y=4]{f}
\Vertex[x=4,y=2]{g}
\Vertex[x=2,y=0]{h}
\Vertex[x=10,y=2]{i}
\Vertex[x=12,y=4]{j}
\Vertex[x=14,y=2]{k}
\Vertex[x=12,y=0]{l}
%Cliques
\Edge[lw=0.03in](a)(b)
\Edge[lw=0.03in](a)(c)
\Edge[lw=0.03in](a)(d)
\Edge[lw=0.01in](b)(c)
\Edge[lw=0.01in](b)(d)
\Edge[lw=0.01in](c)(d)
\Edge[lw=0.01in](e)(f)
\Edge[lw=0.01in](e)(g)
\Edge[lw=0.01in](e)(h)
\Edge[lw=0.03in](f)(g)
\Edge[lw=0.03in](f)(h)
\Edge[lw=0.01in](g)(h)
\Edge[lw=0.01in](i)(j)
\Edge[lw=0.01in](i)(k)
\Edge[lw=0.01in](i)(l)
\Edge[lw=0.01in](j)(k)
\Edge[lw=0.01in](j)(l)
\Edge[lw=0.01in](k)(l)
%Cliques Connection
\Edge[lw=0.03in](a)(f)
\Edge[lw=0.01in](b)(g)
\Edge[lw=0.01in](f)(i)
\Edge[lw=0.01in](g)(j)

    \end{tikzpicture}
  \end{center}
\end{figure}

\paragraph{}
After step 8, we're in a situation where the algorithm will be stucked in a
dead-loop. The tree rooted at $a$ has the right amount of vertices, and step
concerning this tree won't produce any modification. The tree rooted at $e$
can't expand anymore, because all the adjacent vertices have already belong to
him\footnote{See line 28 of the algorithm} and haven't been reseted\footnote{
See line 47 of the algorithm}.
