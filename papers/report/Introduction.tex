\section{Introduction}
\paragraph{}
The goal of the project was to study in depth the article~\cite{JS94}, which,
although pretends to answer positively to a well-known difficult problem,
surprisingly did not attract a lot of attention in the litterature.

% Implementing the algorithm
\paragraph{}
More precisely, in~\cite{JS94} is proposed a polynomial-time algorithm for
constructing a special kind of vertex-partition of any k-connected graph. Our
very first task was to understand this algorithm. For this purpose, we have had
to implement it in Java, using a framework which contains the implementation
of several classical graph algorithms.
%TODO check classical vs classic

% Testing the algorithm
\paragraph{}
% TODO préciser génération pour k-connexe
We also have had to implement a graph generator in order to test both the
validity and the complexity of the proposed algorithm. This approach allowed
us to test our implementation.


% Reading other articles
\paragraph{}
In order to have an entire comprehension of this article, we had to search,
read and understand several related articles.

% Counter example
\paragraph{}
Our implementation has illustrated the fact that specific executions of the
algorithm were leading to an endless loop. With this counter-example, we
proved that the algorithm proposed in~\cite{JS94} has not the expected
complexity.

% Improvement : no endless loop
\paragraph{}
We propose a slight modification to this algorithm, ensuring that instead of
falling in an endless loop, the algorithm will end with a fail state. This
change makes this algorithm easier to use as a base for a new Las-Vegas
algorithm.
