\section{Introduction}
% What we had to do
\paragraph{}
The goal of the project was to study in depth the article~\cite{JS94}, which,
although pretends to answer positively to a well-known difficult problem,
surprisingly did not attract a lot of attention in the litterature.
%% Implementing the algorithm
More precisely, in~\cite{JS94} is proposed a polynomial-time algorithm for
constructing a special kind of vertex-partition of any $k$-connected graph wich 
necessarily exists according to~\cite{GE78,LL77}. %TODO check reference
%% Reading other articles
In order to have an entire comprehension of this article, we had to search,
read and understand several related articles.

% What we have done
\paragraph{}
Our very first task was to understand this algorithm. For this purpose, we have 
had to implement it in Java, using a framework which contains the implementation
of several classic graph algorithms.
%% Testing the algorithm
We also have had to implement a $k$-connected graph generator in order to test both
the validity and the complexity of the proposed algorithm. This approach allowed
us to test our implementation.

% Result
%% Counter example
\paragraph{}
Our implementation has pointed out the fact that specific executions of the
algorithm lead to an endless loop. With those counter-examples, we
proved that the algorithm proposed in~\cite{JS94} has not the expected
complexity and also is not correct.
%% Improvement : no endless loop
We propose a slight modification to this algorithm, ensuring that instead of
falling in an endless loop, the algorithm will end with a fail state. This
change makes this algorithm easier to use as a base for a new Las-Vegas
algorithm.
