\section{Work}

\subsection{Generating random graphs}

\subsection{Calculating a sparse spanning subgraph}
% Necessity of implementing FOREST
\paragraph{}
In order to reduce the execution time of the main algorithm, we had to
implement another algorithm described in~\cite{NaIb92}.

% Algo description
\paragraph{}
This algorithm is called \verb!FOREST! and take a graph $G$ as a parameter. The
output is slightly more complicated since it doesn't simply return the
sparse spanning subgraph. Instead, the algorithm returns an array $E$ of size
$m = |E(G)|$. The graph induced by the set of edges $\bigcup \limits_{i=1}^k E_i$
is $k$-connected, if $k$-connectivity of $G$ is greater or equal to $k$.

% Specific data structure
\paragraph{}
The main difficulty about the implementation of this algorithm was the use of a
specific data structure that allowed an access to the maximum, an insertion and
a deletion with a constant amortized cost. Even if such a structure doesn't
exist, it is possible to create one that has all of the required properties for
this algorithm, since the utilisation is very specific and since an element has
always an integer associated value that can grow only from 1 at each step.

% More details ?

%Tests
\paragraph{}
Tests have been created and runned in order to prove that the result of our
implementation of the \verb!FOREST! algorithm was corresponding to the main
objective:
$$\forall k, k \leq K(G),K(V,\bigcup \limits_{i=1}^k E_i) = k$$

\paragraph{}
This test has been validated on simple graphs, but it has also been seened that
our implementation did not produce the expected result if the input was a
multiple graph. We checked this property by following the algorithm on small
multiple graphs and this showed us that the problem wasn't coming from the
implementation, but really from the algorithm.

\subsection{Implementing the algorithm}

\subsection{Finding a counter-example}
% Not failing on some entries but on some executions
