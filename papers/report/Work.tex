\section{Work}

%TODO find a better title
\subsection{Tools}
\subsubsection{Generating $k$-connected random graphs}
\paragraph{}
In order to test our implementation of the article algorithm, $k$-connected 
graphs had to be generated. Two methods have been implemented, although the
graphs produced are not uniformly randomised, the generation allow us to test
the algorithm on different types of graphs.

\paragraph{}
The first method is based on the vertex degree. To be $k$-connected, a graph
needs to have vertices of degree $k$. For each vertex, edges will be added
until the vertex has a degree $k$. After that, there is a high probability
that the produced graph is $k$-connected.

\paragraph{}
%TODO explain why : complete component so that the graph is localy dense
Since this first method is not deterministic, we tried another method which was
producing a $k$ connected graph by creating several complete components and
linking them together. The algorithm is described in
Figure~\ref{graphGenerator}.

\begin{algorithm}[!h]
  \KwData{$k$ the connectivity,\\ $N$ the number of components,\\ $n$ the size of components}
  \KwResult{A $k$-connected graph}
  $components = \emptyset$\;
  $edges = \emptyset$\;
  \For{i = 1 to N}{
    $components = components \cup newCompleteComponent(n)$
  }
  \For{i = 1 to N-1}{
    \For{j = 1 to k}{
      $edges = edges \cup (components[i][j], components[i+1][j])$
    }
  }
  $g = newGraph(components, edges)$\;
  \Return{$g$}\;
  \caption{\label{graphGenerator}$k$-connected graph generator}
\end{algorithm}

\paragraph{}
First, the complete components are generated. Then $k$ edges are added between the 
first and the second component then $k$ edges are added between the second and the 
third ones and so on.

%TODO explain clearly
\paragraph{}
The edges between every pair of components are not randomly selected. Let say
that the
vertices of each component are numbered from 1 to the size of the component.
The edges between the vertex 1 of the first and the vertex 1 of the second 
component is selected. The same process is repeated for the second vertices to the
$k$-th vertices and then  for the other components. At the end a $k$-connected 
graph is generated.


 
\subsubsection{Calculating a sparse spanning subgraph}
% Necessity of implementing FOREST
\paragraph{}
In order to reduce the execution time of the main algorithm, we had to
implement another algorithm described in~\cite{NaIb92}.

% Algo description
\paragraph{}
This algorithm is called \verb!FOREST! and takes a graph $G=(V,E)$ as a parameter.
The output is slightly more complicated since it does not simply return the
sparse spanning subgraph. Instead, the algorithm returns an array $A$ of set of
edges. The size of the array returned is $|E(G)|$. The graph induced by the
set of edges $\bigcup \limits_{i=1}^k E_i$
is $k$-connected, if $k$-connectivity of $G$ is greater or equal to $k$.

% Specific data structure
\paragraph{}
The main difficulty about the implementation of this algorithm was the use of a
specific data structure that allowed an access to the maximum, an insertion and
a deletion with a constant amortized cost. Even if such a structure doesn't
exist, it is possible to create one that has all of the required properties for
this algorithm, since the utilisation is very specific and since an element has
always an integer associated value that can grow only from 1 at each step.

% More details ?

%Tests
\paragraph{}
Tests have been created and runned in order to prove that the result of our
implementation of the \verb!FOREST! algorithm was corresponding to the main
objective:
$$\forall k, k \leq K(G),K(V,\bigcup \limits_{i=1}^k E_i) = k$$

\paragraph{}
This test has been validated on simple graphs, but it has also been seened that
our implementation did not produce the expected result if the input was a
multiple graph. We checked this property by following the algorithm on small
multiple graphs and this showed us that the problem wasn't coming from the
implementation, but really from the algorithm.

\subsection{Implementing the algorithm}
% Separation in function + misplaced lines
\paragraph{}
In order to facilitate the comprehension of the algorithm implementation and
the debugging, we choose to present the different parts of the algorithm. While
we were implementing, we also noticed that some lines where misplaced, breaking
the coherency of the algorithm.

% Choosing the appropriated data structures
\paragraph{}
In order to have the same complexity as the one proved in the paper, we sticked
to the same data structures. In order to accept graphs where vertices are not
only integers, we added an HashMap, allowing an access in constant time to the
index of a vertex.

% Tree Manipulation
\paragraph{}
In the library we were using, spanning graphs weren't designed to be modified. We
created our own basic class which was handling trees with the needed methods
like cutting the tree on a specific vertex.

\subsection{Criticism}
\subsubsection{Finding a counter-example}
% Not failing on some entries but on some executions
\paragraph{}
Once the main issues of our implementation of the algorithm were solved, we
had still sometimes issues with the algorithm falling in endless loops. Our
investigations on this specific problem ended with the fact that the problem was
coming from the algorithm and not from the implementation. This was ensured by
verifying the validity of the execution of our algorithm step by step.

\label{counter-example}
\subsubsection{Detailed example of a failed execution}

%TODO try to define a macro for figures

\paragraph{}
We will present here an example of an execution that fails with the algorithm
described in~\cite{JS94}. The input of the algorithm is as follows~:
\begin{itemize}
\item $G$ is the graph drawn in Figure~\ref{FE_K2_init}
\item $k = 2$
\item $\mathrm{roots} = \{ e, a \}$
\item $\mathrm{partitionSize} = \{ 5, 7\}$
\end{itemize}

% an array of execution should be given and different colors for components
\paragraph{}
In order to illustrate the status of the algorithm. Vertex and edges which are
included in a tree have a stronger thickness.

\begin{figure}[H]
  \caption{\label{FE_K2_init}Failed Execution : Initial graph}
  \begin{center}
    \begin{tikzpicture}[scale=0.9,transform shape]
      % Vertices
\SetVertexNormal[MinSize=25pt,LineColor=violet,LineWidth=3pt]%V2
\Vertex[x=4.8,y=5  ]{a}
\SetVertexNormal[MinSize=25pt,LineColor=orange,LineWidth=3pt]%V1
\Vertex[x=0  ,y=0  ]{e}
\SetVertexNormal[MinSize=25pt,LineColor=black,LineWidth=1pt]%Vfree
\Vertex[x=3.2,y=8.2]{b}
\Vertex[x=4.8,y=7  ]{c}
\Vertex[x=6.4,y=8.2]{d} 
\Vertex[x=3.2,y=0  ]{f}
\Vertex[x=1.6,y=3.2]{g}
\Vertex[x=1.6,y=1.2]{h}
\Vertex[x=6.4,y=0  ]{i}
\Vertex[x=8  ,y=3.2]{j}
\Vertex[x=9.6,y=0  ]{k}
\Vertex[x=8  ,y=1.2]{l}
%Cliques
\Edge[lw=1pt,color=black](a)(b)%Efree
\Edge[lw=1pt,color=black](a)(c)%Efree
\Edge[lw=1pt,color=black](a)(d)%Efree
\Edge[lw=1pt,color=black](b)(c)%Efree
\Edge[lw=1pt,color=black](b)(d)%Efree
\Edge[lw=1pt,color=black](c)(d)%Efree
\Edge[lw=1pt,color=black](e)(f)%Efree
\Edge[lw=1pt,color=black](e)(g)%Efree
\Edge[lw=1pt,color=black](e)(h)%Efree
\Edge[lw=1pt,color=black](f)(g)%Efree
\Edge[lw=1pt,color=black](f)(h)%Efree
\Edge[lw=1pt,color=black](g)(h)%Efree
\Edge[lw=1pt,color=black](i)(j)%Efree
\Edge[lw=1pt,color=black](i)(k)%Efree
\Edge[lw=1pt,color=black](i)(l)%Efree
\Edge[lw=1pt,color=black](j)(k)%Efree
\Edge[lw=1pt,color=black](j)(l)%Efree
\Edge[lw=1pt,color=black](k)(l)%Efree
%Cliques Connection
\Edge[lw=1pt,color=black](a)(f)%Efree
\Edge[lw=1pt,color=black](b)(g)%Efree
\Edge[lw=1pt,color=black](f)(i)%Efree
\Edge[lw=1pt,color=black](g)(j)%Efree

    \end{tikzpicture}

    \begin{tabular}{|c|c|c|c|c|c|c|c|c|c|c|c|}
\hline
a & b & c & d & e & f & g & h & i & j & k & l\\
\hline
$T_a$ & & & & $T_e$ & & & & & & &\\
\hline
    \end{tabular}
  \end{center}
\end{figure}

\paragraph{}
During the first two steps, each partition needs to acquire more vertices, so
each one will grow by adding all the available neighborhood.

\begin{figure}[H]
  \caption{Failed Execution : Step 1}
  \begin{center}
    \begin{tikzpicture}[scale=0.9,transform shape]
      % Vertices
\SetVertexNormal[LineWidth=3pt]%V2
\Vertex[x=4.8,y=5  ]{a}
\SetVertexNormal[LineWidth=3pt]%V1
\Vertex[x=0  ,y=0  ]{e}
\Vertex[x=3.2,y=0  ]{f}
\Vertex[x=1.6,y=3.2]{g}
\Vertex[x=1.6,y=1.2]{h}
\SetVertexNormal[LineWidth=1pt]%Vfree
\Vertex[x=3.2,y=8.2]{b}
\Vertex[x=4.8,y=7  ]{c}
\Vertex[x=6.4,y=8.2]{d}
\Vertex[x=6.4,y=0  ]{i}
\Vertex[x=8  ,y=3.2]{j}
\Vertex[x=9.6,y=0  ]{k}
\Vertex[x=8  ,y=1.2]{l}
%Cliques
\Edge[lw=1pt](a)(b)
\Edge[lw=1pt](a)(c)
\Edge[lw=1pt](a)(d)
\Edge[lw=1pt](b)(c)
\Edge[lw=1pt](b)(d)
\Edge[lw=1pt](c)(d)
\Edge[lw=3pt](e)(f)
\Edge[lw=3pt](e)(g)
\Edge[lw=3pt](e)(h)
\Edge[lw=1pt](f)(g)
\Edge[lw=1pt](f)(h)
\Edge[lw=1pt](g)(h)
\Edge[lw=1pt](i)(j)
\Edge[lw=1pt](i)(k)
\Edge[lw=1pt](i)(l)
\Edge[lw=1pt](j)(k)
\Edge[lw=1pt](j)(l)
\Edge[lw=1pt](k)(l)
%Cliques Connection
\Edge[lw=1pt](a)(f)
\Edge[lw=1pt](b)(g)
\Edge[lw=1pt](f)(i)
\Edge[lw=1pt](g)(j)

    \end{tikzpicture}
    \begin{tabular}{|c|c|c|c|c|c|c|c|c|c|c|c|}
\hline
a & b & c & d & e & f & g & h & i & j & k & l\\
\hline
$T_a$ & & & & $T_e$ & $T_e$ & $T_e$ & $T_e$ & & & &\\
\hline
    \end{tabular}
  \end{center}
\end{figure}

\begin{figure}[H]
  \caption{Failed Execution : Step 2}
  \begin{center}
    \begin{tikzpicture}[scale=0.9,transform shape]
      % Vertices
\SetVertexNormal[MinSize=25pt,LineColor=violet,LineWidth=3pt]%V2
\Vertex[x=4.8,y=5  ]{a} 
\Vertex[x=3.2,y=8.2]{b}
\Vertex[x=4.8,y=7  ]{c}
\Vertex[x=6.4,y=8.2]{d} 
\SetVertexNormal[MinSize=25pt,LineColor=orange,LineWidth=3pt]%V1
\Vertex[x=0  ,y=0  ]{e}
\Vertex[x=3.2,y=0  ]{f}
\Vertex[x=1.6,y=3.2]{g}
\Vertex[x=1.6,y=1.2]{h}
\SetVertexNormal[MinSize=25pt,LineColor=black,LineWidth=1pt]%Vfree
\Vertex[x=6.4,y=0  ]{i}
\Vertex[x=8  ,y=3.2]{j}
\Vertex[x=9.6,y=0  ]{k}
\Vertex[x=8  ,y=1.2]{l}
%Cliques
\Edge[lw=3pt,color=violet](a)(b)%E2
\Edge[lw=3pt,color=violet](a)(c)%E2
\Edge[lw=3pt,color=violet](a)(d)%E2
\Edge[lw=1pt,color=black](b)(c)%Efree
\Edge[lw=1pt,color=black](b)(d)%Efree
\Edge[lw=1pt,color=black](c)(d)%Efree
\Edge[lw=3pt,color=orange](e)(f)%E1
\Edge[lw=3pt,color=orange](e)(g)%E1
\Edge[lw=3pt,color=orange](e)(h)%E1
\Edge[lw=1pt,color=black](f)(g)%Efree
\Edge[lw=1pt,color=black](f)(h)%Efree
\Edge[lw=1pt,color=black](g)(h)%Efree
\Edge[lw=1pt,color=black](i)(j)%Efree
\Edge[lw=1pt,color=black](i)(k)%Efree
\Edge[lw=1pt,color=black](i)(l)%Efree
\Edge[lw=1pt,color=black](j)(k)%Efree
\Edge[lw=1pt,color=black](j)(l)%Efree
\Edge[lw=1pt,color=black](k)(l)%Efree
%Cliques Connection
\Edge[lw=1pt,color=black](a)(f)%Efree
\Edge[lw=1pt,color=black](b)(g)%Efree
\Edge[lw=1pt,color=black](f)(i)%Efree
\Edge[lw=1pt,color=black](g)(j)%Efree

    \end{tikzpicture}
    \begin{tabular}{|c|c|c|c|c|c|c|c|c|c|c|c|}
\hline
a & b & c & d & e & f & g & h & i & j & k & l\\
\hline
$T_a$ & $T_a$ & $T_a$ & $T_a$ & $T_e$ & $T_e$ & $T_e$ & $T_e$ & & & &\\
\hline
    \end{tabular}
  \end{center}
\end{figure}

\paragraph{}
At step 3, two vertices will be available to increase the size of the current
working tree: $i$ and $j$. Since it needs only one of the two vertices, say
$j$,  is chosen.

\begin{figure}[H]
  \caption{Failed Execution : Step 3}
  \begin{center}
    \begin{tikzpicture}[scale=0.9,transform shape]
      % Vertices
\Vertex[x=7,y=5]{a} 
\Vertex[x=5,y=7]{b}
\Vertex[x=7,y=9]{c}
\Vertex[x=9,y=7]{d} 
\Vertex[x=0,y=2]{e}
\Vertex[x=2,y=4]{f}
\Vertex[x=4,y=2]{g}
\Vertex[x=2,y=0]{h}
\Vertex[x=10,y=2]{i}
\Vertex[x=12,y=4]{j}
\Vertex[x=14,y=2]{k}
\Vertex[x=12,y=0]{l}
%Cliques
\Edge[lw=3pt](a)(b)
\Edge[lw=3pt](a)(c)
\Edge[lw=3pt](a)(d)
\Edge[lw=1pt](b)(c)
\Edge[lw=1pt](b)(d)
\Edge[lw=1pt](c)(d)
\Edge[lw=3pt](e)(f)
\Edge[lw=3pt](e)(g)
\Edge[lw=3pt](e)(h)
\Edge[lw=1pt](f)(g)
\Edge[lw=1pt](f)(h)
\Edge[lw=1pt](g)(h)
\Edge[lw=1pt](i)(j)
\Edge[lw=1pt](i)(k)
\Edge[lw=1pt](i)(l)
\Edge[lw=1pt](j)(k)
\Edge[lw=1pt](j)(l)
\Edge[lw=1pt](k)(l)
%Cliques Connection
\Edge[lw=1pt](a)(f)
\Edge[lw=1pt](b)(g)
\Edge[lw=1pt](f)(i)
\Edge[lw=3pt](g)(j)

    \end{tikzpicture}
    \begin{tabular}{|c|c|c|c|c|c|c|c|c|c|c|c|}
\hline
a & b & c & d & e & f & g & h & i & j & k & l\\
\hline
$T_a$ & $T_a$ & $T_a$ & $T_a$ & $T_e$ & $T_e$ & $T_e$ & $T_e$ & & $T_e$ & &\\
\hline
    \end{tabular}
  \end{center}
\end{figure}

\paragraph{}
At step 4, the tree rooted at $a$ can't be simply extended, since every
adjacent vertex belongs to the other tree. The adjacent vertices are $f$ and
$g$. Both respect the condition necessary to allow the swaping. According to
the algorithm, the vertex to swap must have the lowest degree in the concerned
tree among the swappable candidates. It can't be $g$ since it has a higher
internal degree than $f$ in the concerned tree. The swapped vertex will then be
$f$.

\begin{figure}[H]
  \caption{Failed Execution : Step 4}
  \begin{center}
    \begin{tikzpicture}[scale=0.9,transform shape]
      % Vertices
\SetVertexNormal[LineWidth=3pt]%V2
\Vertex[x=4.8,y=5  ]{a} 
\Vertex[x=3.2,y=8.2]{b}
\Vertex[x=4.8,y=7  ]{c}
\Vertex[x=6.4,y=8.2]{d}
\Vertex[x=3.2,y=0  ]{f} 
\SetVertexNormal[LineWidth=3pt]%V1
\Vertex[x=0  ,y=0  ]{e}
\Vertex[x=1.6,y=3.2]{g}
\Vertex[x=1.6,y=1.2]{h}
\Vertex[x=8  ,y=3.2]{j}
\SetVertexNormal[LineWidth=1pt]%Vfree
\Vertex[x=6.4,y=0  ]{i}
\Vertex[x=9.6,y=0  ]{k}
\Vertex[x=8  ,y=1.2]{l}
%Cliques
\Edge[lw=3pt](a)(b)%E2
\Edge[lw=3pt](a)(c)%E2
\Edge[lw=3pt](a)(d)%E2
\Edge[lw=1pt](b)(c)%Efree
\Edge[lw=1pt](b)(d)%Efree
\Edge[lw=1pt](c)(d)%Efree
\Edge[lw=1pt](e)(f)%Efree
\Edge[lw=3pt](e)(g)%E1
\Edge[lw=3pt](e)(h)%E1
\Edge[lw=1pt](f)(g)%Efree
\Edge[lw=1pt](f)(h)%Efree
\Edge[lw=1pt](g)(h)%Efree
\Edge[lw=1pt](i)(j)%Efree
\Edge[lw=1pt](i)(k)%Efree
\Edge[lw=1pt](i)(l)%Efree
\Edge[lw=1pt](j)(k)%Efree
\Edge[lw=1pt](j)(l)%Efree
\Edge[lw=1pt](k)(l)%Efree
%Cliques Connection
\Edge[lw=3pt](a)(f)%E2
\Edge[lw=1pt](b)(g)%Efree
\Edge[lw=1pt](f)(i)%Efree
\Edge[lw=3pt](g)(j)%E1

    \end{tikzpicture}
    \begin{tabular}{|c|c|c|c|c|c|c|c|c|c|c|c|}
\hline
a & b & c & d & e & f & g & h & i & j & k & l\\
\hline
$T_a$ & $T_a$ & $T_a$ & $T_a$ & $T_e$ & $T_e, T_a$ & $T_e$ & $T_e$ & & $T_e$ & &\\
\hline
    \end{tabular}
  \end{center}
\end{figure}

\paragraph{}
At step 5, the tree rooted at $e$ must get one vertex back, since the other
tree has taken one of it's vertices. The two candidates are $i$ and $k$. Since
only one is missing, only one must be added. In our case, $i$ will be added.

\begin{figure}[H]
  \caption{Failed Execution : Step 5}
  \begin{center}
    \begin{tikzpicture}[scale=0.9,transform shape]
      % Vertices
\SetVertexNormal[MinSize=25pt,LineColor=violet,LineWidth=3pt]%V2
\Vertex[x=4.8,y=5  ]{a} 
\Vertex[x=3.2,y=8.2]{b}
\Vertex[x=4.8,y=7  ]{c}
\Vertex[x=6.4,y=8.2]{d} 
\Vertex[x=3.2,y=0  ]{f}
\SetVertexNormal[MinSize=25pt,LineColor=orange,LineWidth=3pt]%V1
\Vertex[x=0  ,y=0  ]{e}
\Vertex[x=1.6,y=3.2]{g}
\Vertex[x=1.6,y=1.2]{h}
\Vertex[x=6.4,y=0  ]{i}
\Vertex[x=8  ,y=3.2]{j}
\SetVertexNormal[MinSize=25pt,LineColor=black,LineWidth=1pt]%Vfree
\Vertex[x=9.6,y=0  ]{k}
\Vertex[x=8  ,y=1.2]{l}
%Cliques
\Edge[lw=3pt,color=violet](a)(b)%E2
\Edge[lw=3pt,color=violet](a)(c)%E2
\Edge[lw=3pt,color=violet](a)(d)%E2
\Edge[lw=1pt,color=black](b)(c)%Efree
\Edge[lw=1pt,color=black](b)(d)%Efree
\Edge[lw=1pt,color=black](c)(d)%Efree
\Edge[lw=1pt,color=black](e)(f)%Efree
\Edge[lw=3pt,color=orange](e)(g)%E1
\Edge[lw=3pt,color=orange](e)(h)%E1
\Edge[lw=1pt,color=black](f)(g)%Efree
\Edge[lw=1pt,color=black](f)(h)%Efree
\Edge[lw=1pt,color=black](g)(h)%Efree
\Edge[lw=3pt,color=orange](i)(j)%E1
\Edge[lw=1pt,color=black](i)(k)%Efree
\Edge[lw=1pt,color=black](i)(l)%Efree
\Edge[lw=1pt,color=black](j)(k)%Efree
\Edge[lw=1pt,color=black](j)(l)%Efree
\Edge[lw=1pt,color=black](k)(l)%Efree
%Cliques Connection
\Edge[lw=3pt,color=violet](a)(f)%E2
\Edge[lw=1pt,color=black](b)(g)%Efree
\Edge[lw=1pt,color=black](f)(i)%Efree
\Edge[lw=3pt,color=orange](g)(j)%E1

    \end{tikzpicture}
    \begin{tabular}{|c|c|c|c|c|c|c|c|c|c|c|c|}
\hline
a & b & c & d & e & f & g & h & i & j & k & l\\
\hline
$T_a$ & $T_a$ & $T_a$ & $T_a$ & $T_e$ & $T_e, T_a$ & $T_e$ & $T_e$ & $T_e$ & $T_e$ & &\\
\hline
    \end{tabular}
  \end{center}
\end{figure}

\paragraph{}
At step 6, the tree rooted at $a$ has no available adjacent vertex, it then
has to swap a vertex among the neighborhood: $e$,$g$,$h$,$i$. $e$ is not an
option since it's the root of a partition. $g$ has an internal degree of 2, it
can't be choosen since $h$ and $i$ have an internal degree of 1. The swapped
vertex must be $h$ or $i$. Since nothing in the algorithm determines which one
must be choosen, $h$ might be swapped at that step.

\begin{figure}[H]
  \caption{Failed Execution : Step 6}
  \begin{center}
    \begin{tikzpicture}[scale=0.9,transform shape]
      \input{graphTikZ/failedK2/step6.tex}
    \end{tikzpicture}
    \begin{tabular}{|c|c|c|c|c|c|c|c|c|c|c|c|}
\hline
a & b & c & d & e & f & g & h & i & j & k & l\\
\hline
$T_a$ & $T_a$ & $T_a$ & $T_a$ & $T_e$ & $T_e, T_a$ & $T_e$ & $T_e, T_a$ & $T_e$ & $T_e$ & &\\
\hline
    \end{tabular}
  \end{center}
\end{figure}

\paragraph{}
At step 7, the tree rooted at $e$ must add a vertex among $k$ and $l$, the
vertex $k$ can be freely added with the edge $\{i,k\}$.

\begin{figure}[H]
  \caption{Failed Execution : Step 7}
  \begin{center}
    \begin{tikzpicture}[scale=0.9,transform shape]
      \input{graphTikZ/failedK2/step7.tex}
    \end{tikzpicture}
    \begin{tabular}{|c|c|c|c|c|c|c|c|c|c|c|c|}
\hline
a & b & c & d & e & f & g & h & i & j & k & l\\
\hline
$T_a$ & $T_a$ & $T_a$ & $T_a$ & $T_e$ & $T_e, T_a$ & $T_e$ & $T_e, T_a$ & $T_e$ & $T_e$ & $T_e$ &\\
\hline
    \end{tabular}
  \end{center}
\end{figure}

\paragraph{}
At step 8, the tree rooted at $a$ must take a last vertex, it has no available
adjacent vertex, it must then take a vertex from the tree rooted at $e$. The
swappable vertices are $g$ and $i$. Since they both have the
same internal degree in the concerned tree, $g$ can be choosen as a valid
choice for the extension.

\paragraph{}
By swapping with $g$, the subtree rooted at $g$ must be cut from the tree
rooted at $e$. All the removed vertices ($i$,$j$,$k$) will be set back as
unexplored, but not the swapped vertex, $g$.

\begin{figure}[H]
  \caption{Failed Execution : Step 8}
  \begin{center}
    \begin{tikzpicture}[scale=0.9,transform shape]
      \input{graphTikZ/failedK2/step8.tex}
    \end{tikzpicture}
    \begin{tabular}{|c|c|c|c|c|c|c|c|c|c|c|c|}
\hline
a & b & c & d & e & f & g & h & i & j & k & l\\
\hline
$T_a$ & $T_a$ & $T_a$ & $T_a$ & $T_e$ & $T_e, T_a$ & $T_e,T_a$ & $T_e, T_a$ & & &  &\\
\hline
    \end{tabular}
  \end{center}
\end{figure}

\paragraph{}
After step 8, we're in a situation where the algorithm will be stucked in a
dead-loop. The tree rooted at $a$ has the right amount of vertices, and step
concerning this tree won't produce any modification. The tree rooted at $e$
can't expand anymore, because all the adjacent vertices have already belong to
him\footnote{See line 28 of the algorithm} and haven't been reseted\footnote{
See line 47 of the algorithm}.


\subsubsection{Counter-example in general case}
\paragraph{}
In order to ensure that $k=2$ is not a specific case where the algorithm
presented in~\cite{JS94} is not determinist, we will present a proof that an
execution can loop endlessly for any $k$.

\paragraph{}
The proof we will present is a proof by induction. Since we already exhibited
a counter-example for $k=2$, we only need to proove the recurrence.

\paragraph{}
Let $G$ be a $k$-connex graph for which the execution of the algorithm has a
probability of looping endlessly strictly greater than $0$ for the entry
$\{r_1,r_2, \dots, r_k\}$ and $\{n_1,n_2, \dots, n_k\}$. Then a graph $G'$ can
be found such as it is $k+1$-connex and it exists an entry for which the
execution of the algorithm has a probability of looping endlessly strictly
greater than $0$.

\paragraph{Proof:}
If we create $G'$ by adding a new vertex $v$ connected to every vertex of $G$,
i.e. $V(G') = V(G) \cup v$ and
$E(G') = E(G) \cup \bigcup \limits_{u \in V(G)} \{u,v\}$, it is obvious that
$G'$ will be $k+1$-connex, since every minimal vertex cut of $G$ needs to have
 one vertex more in order to be a cut of $G'$.

\paragraph{}
An execution of the algorithm with $G'$, $\{r_1,r_2, \dots, r_k, v\}$,
$\{n_1,n_2, \dots, n_k, 1$ as input will have a probability of looping
endlessly strictly greater than $0$, because the $v$ introduced has no influence
since the associated tree will be full since the beginning and the new edges
won't be useable either.
% Bad presentation ...
\subsection{Improvements}
\subsubsection{Detecting endless loops}
\paragraph{}
A counter example that shows the algorithm can fall in endless loops have been
exhibited. This raise one question : is it possible to detect when there is an
endless loop? An improvement has been added to the algorithm to exit the
execution when the algorithm enters in an endless loop.

\paragraph{}
To do so, we save the last root index where we have added or swapped a vertex. If
the saved index is equal to the current root index and no vertex have been
added or swap then it means that the algorithm has entered in an endless loop. Then
the execution is stopped.

\subsubsection{Vertex selection condition}
%TODO 
