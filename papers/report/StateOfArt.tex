\section{State of Art}
\subsection{Computing $k$-connectivity}
\paragraph{}
As we have seen in the previous section, the connectivity of a graph is equal to
the size of its minimal vertex cut.
To do so the max-flow min-cut theorem states that the maximun flow between to
vertices is equal to the minimun cut to disconect this to vertices.
But if we use directly a flow algorithm, we will compute the edge-connectivity
and not the vertex-connectivity.
%TODO introduce edge-connectivity before

So first the graph need to be transform. We need capacity on the nodes. To do so
each nodes will be divided into two new nodes. Each incoming edge will be
attached to the first node and each leaving edge to the second node. Another
edge will be added between the two new nodes.
The capacity of all edges of the graph will be set to 1.

Then to find the connectivity, we compute the maximum flow between each pair of
nodes and we keep only the lowest value. This value is the connectivity.


\begin{algorithm}[!h]
    \KwData{$G=(V,E)$ a graph}
    \KwResult{$k$ the connecitvity}
    $min = \infty$\;
    $G^{'} = TransformNode(G)$\;
    \ForAll{$u,v \in V, u \neq v, (u,v) \notin E$}{
        $f = maximumFlow(G^{'},u,v)$\;
        \If{$f < min$}{
            $min = f$\;
    }
}
    \Return{$min$}\;
    \caption{Compute the connectivity}
\end{algorithm}

\subsection{k-partitioning}
\paragraph{}
In this section, we are going to focus on the differents existing
$k$-partitioning algorithms.

\paragraph{}
First of all, for a general graph $G$, finding a $k$-partition is a $NP$-hard
problem~\cite{Dyer1985139}. No general polynomial algorithms has yet been found.
So we are going to focus on particular graphs.

\paragraph{}
It has been proven that for a $k$-connex graph G a $k$-partition
exists~\cite{GE78,LL77}.
According to~\cite{JS94}, there are polynomial solutions for
$k=2$~\cite{GE78,LL77} and $k=3$.

\paragraph{}
More recently, it has been proven in~\cite{Nakano1997315} that if $G$ is a planar graph, there is a linear-time algorithm for $k = 4$.




