In this section, existing algorithms related to
$k$-connectivity and $k$-partitioning are presented.

\subsubsection{Computing $k$-connectivity}
\paragraph{}
As we have seen in section 2, the connectivity of a graph is equal to
the size of its minimal vertex cut. The max-flow min-cut theorem states
that the maximun flow between to vertices is equal to the minimum cut to 
disconnect these two vertices. But if we use flow algorithm directly, 
we will compute the edge-connectivity and not the vertex-connectivity.
%TODO introduce edge-connectivity before

So first the graph needs to be transformed. We need a capacity on the nodes.
To do so each nodes will be divided into two new nodes. Each incoming edge will
be attached to the first node and each leaving edge to the second node. Another
edge will be added between the two new nodes.
The capacity of all edges of the graph will be set to 1.

Then to find the connectivity, we compute the maximum flow between each pair of
nodes and we keep only the lowest value. This value is the connectivity.


\begin{algorithm}[!h]
    \KwData{$G=(V,E)$ a graph}
    \KwResult{$k$ the connectivity}
    $min = \infty$\;
    $G^{'} = TransformNode(G)$\;
    \ForAll{$u,v \in V, u \neq v, (u,v) \notin E$}{
        $f = maximumFlow(G^{'},u,v)$\;
        \If{$f < min$}{
            $min = f$\;
    }
}
    \Return{$min$}\;
    \caption{Computing the connectivity of a graph}
\end{algorithm}

\subsubsection{k-partitioning}
%TODO Check all ref and correct error
\paragraph{}
In this section, we focus on the different existing
$k$-partitioning algorithms.

\paragraph{}
First of all, for a general graph $G$, finding a $k$-partition is a $NP$-hard
problem~\cite{Dyer1985139}. No general polynomial algorithms has yet been found.
So we are going to focus on particular graphs.

\paragraph{}
It has been proven that for a $k$-connected graph G a $k$-partition
exists~\cite{GE78,LL77}.
According to~\cite{JS94}, there are polynomial solutions for
$k=2$~\cite{GE78,LL77} and $k=3$.

\paragraph{}
More recently, it has been proven in~\cite{Nakano1997315} that if $G$ is a planar graph, there is a linear-time algorithm for $k = 4$.




