\subsection{State of the art: existing algorithms}
In this section, we present algorithms related to
$k$-connectivity and $k$-partitioning.

\subsubsection{Computing $k$-connectivity}
\paragraph{}
As we have seen in Section~\ref{defConnectivity}, the connectivity of a graph is equal to
the size of its minimal vertex cut. The max-flow min-cut Theorem states
that the maximum flow between to vertices is equal to the minimum cut to 
disconnect these two vertices. But if we use flow algorithm directly, 
do not compute the connectivity, since it will be possible to have two different
flows passing through the same vertex.
%TODO introduce edge-connectivity before if we want to use it

\paragraph{}
So first the graph needs to be transformed. For this purpose, we need a capacity
on the vertices.
To do so each vertex is divided into two new vertices. Each incoming edge is
attached to the first vertex and each leaving edge to the second vertex. Another
edge is added between the two new vertices.
The capacity of all edges of the graph is set to 1.

\paragraph{}
Then to find the connectivity, we compute the maximum flow between each pair of
nodes and we keep only the lowest value. This value is the connectivity.


\begin{algorithm}[!h]
    \KwData{$G=(V,E)$ a graph}
    \KwResult{$k$ the connectivity}
    $min = \infty$\;
    $G^{'} = TransformNode(G)$\;
    \ForAll{$u,v \in V, u \neq v, (u,v) \notin E$}{
        $f = maximumFlow(G^{'},u,v)$\;
        \If{$f < min$}{
            $min = f$\;
    }
}
    \Return{$min$}\;
    \caption{Computing the connectivity of a graph}
\end{algorithm}
%TODO ideally, place an example here

\subsubsection{k-partitioning}
%TODO Check all ref and correct error
\paragraph{}
In this section, we focus on the different existing
$k$-partitioning algorithms.

\paragraph{}
First of all, since it has been proven that for some kinds of graphs finding a
$k$-partition is an $NP$-hard problem~\cite{BF06}, finding a partition in a
graph is $NP$-hard in general.

\paragraph{}
As we mentioned earlier, it has been proven that, for a $k$-connected graph
$G$, a $k$-partition always exists~\cite{GE78,LL77}. The algorithm
from~\cite{JS94} apart, no polynomial general algorithm, i.e. for all values of
$k$, have been proposed. Polynomial algorithms have been proposed for
$k=2$\cite{Suzuki1990227} and $k=3$\cite{Wada1994}. More recently, it has
been proven in~\cite{Nakano1997315} that if $G$ is a planar graph and if the
roots of the partition are located on the same face of a plane embedding of $G$,
there is a linear-time algorithm for $k = 4$.


