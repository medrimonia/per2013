\section{Conclusion}

% First: the principal : proof that the algorithm doesn't work
\paragraph{}
Through this study, we have demonstrated that the algorithm proposed
in~\cite{JS94} is not correct, for any $k$. The private communication from E.
Gy{\H o}ri mentioned in~\cite{Nakano1997315} has been prooved with a
counter-example, exhibiting the possibility of the algorithm to fall into
endless loops. Then the property was extended to any $k$ with an inductive
proof.

% How we found the counter example
\paragraph{}
The counter-example was found by implementing the algorithm and running it on
randomly generated graphs. Therefore, the tools implemented can be used to
test new approaches similars to the algorithm or needing the generation of
$k$-connected graphs.

% Situation now, solved for some k but not for any k
\paragraph{}
Since we prooved that the algorithm proposed in~\cite{JS94} is not correct,
there is still no polynomial solutions that solve the $k$-partition problem
for any $k$. Even if algorithms have been proposed which find a solution to
this problem for specific value of $k$ or for specific kinds of graph.

% Las-Vegas
\paragraph{}
We proposed a modification which allows to avoid endless loops, informing of
a failure if the execution would have lead to an endless loops. This specific
modification could allow to create a Las Vegas algorithm if the probability
that the algorithm returns successfully the solution is equal or greater than 
$\frac{1}{2}$. Prooving this would not bring a polynomial solution to the
$k$-partition problem, but it would already proove that the $k$-partition
problem is in $ZPP$.