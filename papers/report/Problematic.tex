\subsection{Problem description}
\paragraph{}
The algorithm proposed in the article~\cite{JS94} claims to be able to find
a $k$-partition in a $k$-connected graph in a polynomial time, and this no
matter what is $k$. The complexity of the claimed algorithm for a graph
$G = (V,E)$ with $|V| = n$ is $O(k^2 n^2)$.

\paragraph{}
The $k$-partition problem which the article pretends to resolve can be
formally defined as follows. Given a $k$-connected graph $G = (V,E)$, a set of
$k$ vertices $R = \{r_1, r_2, \dots, r_k \}$ and $k$ integers
$S = \{s_1,s_2, \dots, s_i\}$ summing up to the order of $G$. Find a partition
$P = \{p_1, p_2, \dots, p_k \}$ of $V$ such that for any $i$ from $1$ to $k$,
$p_i$ contains $r_i$ and $|p_i| = s_i$. An illustrated example is presented in
Figure~\ref{fig:inputOutput}.

\newcommand{\formalTitle}[1]{\textbf{\large #1}\vspace{0.5cm}}

\begin{figure}[H]
  \caption{\label{fig:inputOutput}Input and output of the $k$-partition
    problem}
  \vspace{0.5cm}
  \begin{minipage}{.5\textwidth}
    \begin{center}
      \formalTitle{Input}\\
      \begin{tikzpicture}[x=.06\textwidth, y=.06\textwidth,transform shape]
        % Vertices
\SetVertexNormal[MinSize=25pt,LineColor=violet,LineWidth=3pt]%V1
\Vertex[x=  3,y=0]{a}
\SetVertexNormal[MinSize=25pt,LineColor=orange,LineWidth=3pt]%V2
\Vertex[x=5.5,y=4]{g}
\SetVertexNormal[MinSize=25pt,LineColor=blue,LineWidth=3pt]%V3
\Vertex[x= 11,y=4]{c}
\SetVertexNormal[MinSize=25pt,LineColor=black,LineWidth=3pt]%Vfree
\Vertex[x=  8,y=0]{b}
\Vertex[x=  8,y=8]{d} 
\Vertex[x=  3,y=8]{e}
\Vertex[x=  0,y=4]{f}

%Cliques
\Edge[lw=1pt,color=black](a)(b)%Efree
\Edge[lw=1pt,color=black](a)(g)%Efree
\Edge[lw=1pt,color=black](b)(c)%Efree
\Edge[lw=1pt,color=black](b)(g)%Efree
\Edge[lw=1pt,color=black](c)(d)%Efree
\Edge[lw=1pt,color=black](c)(g)%Efree
\Edge[lw=1pt,color=black](d)(e)%Efree
\Edge[lw=1pt,color=black](d)(g)%Efree
\Edge[lw=1pt,color=black](e)(f)%Efree
\Edge[lw=1pt,color=black](e)(g)%Efree
\Edge[lw=1pt,color=black](f)(a)%Efree
\Edge[lw=1pt,color=black](f)(g)%Efree

      \end{tikzpicture}
    \end{center}
  \end{minipage}
  \begin{minipage}{.5\textwidth}
    \begin{center}
      \formalTitle{Output}\\
      \begin{tikzpicture}[x=.06\textwidth, y=.06\textwidth,transform shape]
        % Vertices
\SetVertexNormal[MinSize=25pt,LineColor=violet,LineWidth=3pt]%V1
\Vertex[x=  3,y=0]{a}
\Vertex[x=  0,y=4]{f}
\SetVertexNormal[MinSize=25pt,LineColor=orange,LineWidth=3pt]%V2
\Vertex[x=5.5,y=4]{g}
\SetVertexNormal[MinSize=25pt,LineColor=blue,LineWidth=3pt]%V3
\Vertex[x= 11,y=4]{c}
\Vertex[x=  8,y=0]{b}
\Vertex[x=  8,y=8]{d} 
\Vertex[x=  3,y=8]{e}
\SetVertexNormal[MinSize=25pt,LineColor=black,LineWidth=3pt]%Vfree

%Cliques
\Edge[lw=1pt,color=black](a)(b)%Efree
\Edge[lw=1pt,color=black](a)(g)%Efree
\Edge[lw=3pt,color=blue](b)(c)%E3
\Edge[lw=1pt,color=black](b)(g)%Efree
\Edge[lw=3pt,color=blue](c)(d)%E3
\Edge[lw=1pt,color=black](c)(g)%Efree
\Edge[lw=3pt,color=blue](d)(e)%E3
\Edge[lw=1pt,color=black](d)(g)%Efree
\Edge[lw=1pt,color=black](e)(f)%Efree
\Edge[lw=1pt,color=black](e)(g)%Efree
\Edge[lw=3pt,color=violet](f)(a)%E1
\Edge[lw=1pt,color=black](f)(g)%Efree

      \end{tikzpicture}
    \end{center}
  \end{minipage}
  \begin{minipage}{.5\textwidth}
    $$k = 3$$
    $$R = \{a,g,c\}$$
    $$S = \{2,1,4\}$$
  \end{minipage}
  \begin{minipage}{.5\textwidth}
    $$P = \big\{\{a, f\},\{g\},\{b, c, d, e\}\big\}$$
  \end{minipage}
\end{figure}

% Is a formal description really necessary
%\newcommand{\formalTitle}[1]{\textbf{\large #1}}
%\begin{figure}[H]
%  \caption{Formal description of the $k$-partition problem}
%  \begin{minipage}{.5\textwidth}
%    \formalTitle{Input}
%    \small
%    \begin{itemize}
%    \item $G = (V,E)$, a $k$-connected graph
%    \item $k$
%    \item $R$, a list of roots wished for the components
%    \item $S$, a list of the size wished for the components
%    \end{itemize}
%  \end{minipage}
%  \begin{minipage}{.5\textwidth}
%    \formalTitle{Output}
%  \end{minipage}
%\end{figure}

\subsection{Algorithm description}
\subsubsection{Principles}
\paragraph{}
The main idea of the algorithm proposed in~\cite{JS94} is to expand different
trees from the roots given as input. The specificities of this expansion is
that it is both {\em progressive} and {\em balanced}.
\paragraph{}
It is {\em progressive} because the algorithm iterates over the different
components. Each component will try to acquire free vertices (vertices which
do not belong to a component) and if there is no free vertices among its
neighborhood, it will try to steal a vertex from another component.
\paragraph{}
The algorithm is {\em balanced} because a component can only steal a vertex
from a component which has a lower ratio of the number of wished vertices over
the number of vertices currently owned by the component.
\paragraph{}
In order to avoid endless loops engendered by two components stealing the same
vertex one after the other, a memory of the components to which a vertex has
belonged is maintained.
\paragraph{}
A proof of the complexity of this algorithm is given in the
article~\cite{JS94}. This proof can be summarized as follows.
\begin{enumerate}
\item The number of iterations of the expansion is $O(kn)$.
\item Every step of an iteration is $O(m)$.
\item An existing algorithm with complexity $O(m)$  allows to
  reduce the number of edges in a $k$-connected graph to a number of edge
  $O(kn)$~\cite{NaIb92}.
\end{enumerate}
\paragraph{}
From those three properties, the authors affirm that the complexity of the
algorithm is $O(k^2n^2)$. But, even if the second possibility can be
verified and if the validity of the third one does nott influence the fact that
this algorithm is polynomial, the proof of the first property seems to be wrong
since we exhibit a counter-example in section~\ref{counter-example}, showing
that the algorithm may loop endlessly.

\subsubsection{Used data}
\paragraph{}
As it was mentioned previously, the algorithm of~\cite{JS94} partition the graph
into trees. Thoses trees needs to be stored, so both vertices and edges of
each tree must be memorized and updated through the execution. The tree rooted
at $v = r_i$ is denoted $T_v$.
\paragraph{}
The history of the trees to which a vertex has belonged is stored in a
set of vertices. An {\em HashMap} with vertices as keys and a set of vertices
as values can then be used to obtain the history of any vertex. The data are
stored in a structure called $\mathrm{treeNode}$ and the set of roots whose
tree has owned the vertex $v$ is denoted $\mathrm{treeNode}[v]$. It is
important to note that the history of some vertices can be reseted in some
specific cases (see section~\ref{cutoff-subtrees}).
\paragraph{}
The feature ensuring that the expansion is balanced is called the $p$-value.
Each tree has its own $p$-value which is defined as the number of vertices
wished for the tree divided by the number of vertices currently in the tree. A
tree which has a $p$-value of 1 is a component which has the right amount
of vertices. When iterating over such a tree, nothing will be done,
because it does not need to acquire more vertices.

\subsubsection{Acquiring vertices}
\paragraph{}
When iterating over a tree which needs to gain vertices, the first step
is to compute the neighborhood of the tree, i.e. the vertices of the
graph which are neighbors to one (or more) of the vertices of the tree
but are not in it. If among this neighborhood, there is at least one
free vertex, i.e. a vertex which does not belong to any tree. The
tree will acquire all the free vertices of this neighborhood, except if
there are more free vertices than required. In that case, the tree
acquires enough vertices so that its order reached the desired size, but no
more. The vertices to acquire are chosen randomly among the available ones.

\subsubsection{Stealing vertices}
% Stealable vertices
\paragraph{}
If there is not any free vertex in the neighborhood of the tree, then it tries
to steal a vertex from another tree. Several rules defines whether a vertex can
be stolen or not.

%TODO : Explain!
\begin{itemize}
\item The root of a tree (i.e. the $r_i$'s) cannot be stolen.
\item A tree $T_v$ cannot steal a vertex $u$ if $T_v$ belongs to
  $\mathrm{treeNode}[u]$.
\item A tree cannot steal a vertex from another tree which has a
  lower $p$-value.
\end{itemize}

% Choosing the component
\paragraph{}
Once those three rules have been applied to remove some vertices of the
neighborhood, if there is no vertex left, then the tree doest not steal
any other vertex during this iteration. On the other side, if there are still
multiple vertices remaining, then one vertex is chosen as the vertex to
steal. The following procedure is applied.
\begin{enumerate}
\item Keep only the vertices which have the smallest $p$-value of the set.
\item Among this new set, keep only the vertices which have the smallest
  degree inside the tree.
\item If there are still multiple vertices, pick one randomly.
\end{enumerate}

\paragraph{\label{cutoff-subtrees}}
If the vertex $v$ which must be stolen has degree greater than 1 in
its tree, removing it disconnects the tree. To avoid this kind of
situation, the subtree rooted at $v$ is cut and only $v$ is
stolen. All the other vertices $u$ of the subtree have their history
cleaned and are considered again as free vertices, i.e.
$\mathrm{treeNode}[u] = \emptyset$.