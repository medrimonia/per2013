\section{State of Art}

\subsection{Undirected graphs}

\paragraph{}
Undirected graphs are mathematical objects composed of a set of vertices $V$
and a set of edges $E$ and are noted $G = (V,E)$.

\paragraph{}
Also called nodes, vertices are mathematical objects which might virtually
represent anything, a vertices is usually noted $v$.

\paragraph{}
Edges are an unordered pair of vertices noted $e = (u,v)$. Edges might
contain twice the same vertex, in this case, the edge is called a {\em loop}.

\subsubsection{Fundamentals notations}

\paragraph{}
The size of a graph is the number of edges it contains and is noted
$m = |E|$.

\paragraph{}
The order of a graph is the number of vertices it contains and is noted
$n = |V|$.

\paragraph{}
The degree of a vertex is equal to the number of edges incident to it, loops
are counted twice.

\paragraph{}
A path of length $n$ is a sequence of alternated vertices and edged, noted
$P = \{v_0, e_1, v_1, e_2, ..., e_n, v_n\}$ such as :
$\forall i \in \{1,2, ..., n\}, e_i = \{v_i, v_{i-1}\}$. If $v_0 = v_n$, $P$ is
a cycle.

\paragraph{}
A graph  is {\em connected} if $\forall u,v \in V$ a path exists from $u$
to $v$. Otherwise, the graph is {\em unconnected}

\subsubsection{Types of graphs}

\paragraph{}
A tree is an undirected graph without cycles,

\subsubsection{Subgraphs}

\subsubsection{Cuts}

\subsubsection{$k$-connectivity}
\paragraph{}
A graph with a least two vertices is k-connected if, for every pair of vertices, there is at least k vertex disjoint paths between these vertices.

\paragraph{}
A graph is k-connected if the smallest subset which disconnect the graph if you delete it is of size k.

